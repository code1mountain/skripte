\documentclass[a4paper]{scrartcl}
\author{Me}
\title{Was für die Prüfungen gemacht werden muss}
\usepackage[utf8]{inputenc} % use utf8 file encoding for TeX sources
\usepackage[T1]{fontenc}    % avoid garbled Unicode text in pdf


\begin{document}
    \section{Analysis}
        \begin{itemize}
            \item Komplexe Zahlen
            \item Funktionen
            \item Differentialrechnung
            \item Integralrechnung
        \end{itemize}
    \section{Lineare Algebra}
        \begin{itemize}
            \item Logik, Digitaltechnik
            \item Mengen
            \item Relationen, Funktionen
            \item Induktion und Rekursion
            \item Lineare Gleichungssysteme, Gauß-Algorithmus
            \item Determinanten
            \item Matrizen
            \item Vektoren, Koordinaten
            \item Vektoren, Geraden und Ebenen
            \item Skalarprodukt
            \item Vektorprodukt
            \item Gruppen, Körper, Vektorräume
            \item Basis und Dimension
            \item Lineare Abbildungen
            \item Eigenwerte, Eigenvektoren, Eigenräume
            \item Komplexe Zahlen
        \end{itemize}
    \section{Netzwerke}
        \begin{itemize}
            \item Netzwerke LAN WAN DNS 
            \item Ethernet Bridging VLAN SPT PPT
            \item IP ICMP TCP UDP
            \item Router Routing
        \end{itemize}
    \section{Programmieren}
        \begin{itemize}
            \item Ein- und Ausgabe: Tastatur und Bildschirm
            \item Zahlen, Berechnungen
            \item Schleifen
            \item Verzweigungen
            \item Entwurf und Dokumentation
            \item Methoden / Funktionen
            \item Sichtbarkeit (Scope) von Variablen
            \item Programmierstil
            \item Arrays und Strings
            \item Objekte definieren, erzeugen, referenzieren
            \item Listen/Mengen von Objekten
            \item Exception Handling
            \item Die Standard-Bibliothek
            \item Rekursive Methoden/Funktionen
        \end{itemize}
    \section{IxD}
        \begin{itemize}
            \item Prototypen vorstellen
        \end{itemize}
\end{document}