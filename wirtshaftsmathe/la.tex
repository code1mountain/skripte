\documentclass[a4paper]{scrartcl}
\author{Me}
\title{Grundlagen der Informatik}
\usepackage[utf8]{inputenc} % use utf8 file encoding for TeX sources
\usepackage[T1]{fontenc}    % avoid garbled Unicode text in pdf
\usepackage[german]{babel}
\usepackage{amsmath}
\begin{document}
    \maketitle
    \newpage
    \tableofcontents
    \newpage

    


    \section{Folgen und Reihen}
    Folgen
        werden durch index angegeben und nicht unbedingt durch funktionswert, oft auch rekursiv oder mit hilfe vom ersten wenn man a_n herausfinden will
    Reihen,
        Sind folgeglieder die bis zu n (oder auch undendlich für undendliche Reihen) auffaddiert werden

    arithmetische Folgen
        an+1 = an + d
        an = a1 +(n-1)d, d e R fest
        Reihe: N(2a_1 + (N-1)d)/2
    geometrische Folge 
        an+1 = an *q, q e R fest
        an = a1 + q^(n-1)
        Reihe = a1 * (q^N -1)/(q-1)
    
    \section{Zinsen}
        Sollen Zinsen nicht mitverzinst werden glt die Formel 
        d = k_0 + i; KN = K0(1+n*i)
        K0 e R+ (Kapital)
        n e R+ , Laufzeit 
        kN Endpaital Kaptial + Zinsen
        i Zinsrate
        i* 100 Zinsrate in Prozent

        Zinseszins
        Kn = K0(1+i)^n

        stetige Verzinsung
        K0 +e ^(i*n)

        gemischte Verszinsung
        Kn = K0(1+i)^ [n](1+i*(n-[n]))
        
        Freigeld
        Besitz von Geld wird verzinst (eher ein Wertverfall da niemand die Zinsgebühren erhält) damit es wie zB. Lebensmittel verdirbt. Dadruch wird der Umlauf erhöht.
        hier Wird der Zinsatz dann nur abgezogen anstatt addiert 

        Investitionsrechnung
        Die Investionsrechnung wird verwendet um zu bestimmen ob sich eine Investions im Laufe der Zei tlohenen wird. Zuerst zieht man die Kosten der Investions vom Kapital ab.
        Gegenfalls schaut man ob es periodische Zahlungen gibt die man leisten muss (Wartungsarbeiten, Kreditszinsen etc) und ziet diese auch ab. Man schätz nun den Gewinn den die Investion im Monat/Jahr etc
        macht, zieht noch eventuelle Zhalungen ab und erhält den Monats/Jahresgewinn. DIes macht man fürr alle Monate für die man eine SChätzung machen kann, verzinst diese dementsprechend und vergleciht den 
        Gesamtgewinn mit dem Gewinn den man erhält wenn man nciht investiert und sein Kapital einfachüber den selben Zeitraum verzinsen lässt.

        Rentenrechnung
        Rente = Regelmäßige Zahlung
        dauer endlich rente, leib rente (rente endet mit tod), ewige rente (Pacht)
        Periode mmonat jahr 
        termin wann wird gezshlt davo
        zinsperiode wann werden zinsen ausgezhalt (glecih mit rente oder wird rente im moant gezahlt aber zinsen im jahr)

        Formel: Rn = r*q*(q^n -1)/(q-1)
        r Ratenzahlung
        n Ratenperioden
        i Zinsatz, q = 1+i
        Rn Enbetrag
        Formel ist für vorschüssige Renten, für nachschüssige gilt:
        r*(q^n -1)/(q-1)

        R0 = Rn/q^n (barwert, also was wird gezahlt wenn alles am anfang direkt 'in Bar' gezahlt wird)

        Ratenperiode < zinsperiode
        Wenn die Ratenperioden ein quaratl beträgt, die Zinsperiode ein Jahr, dannrechnet man vier mal die Ratenperiode damit man auf ein Jahr kommt und rechnet dann auf diesen Wert (r_e genantn) 
        die Zinsen

        re = r+ (m+i/2 + (m+1))
        Rn = re * (q^n -1) /(q-1)


        Tilgungsrechnugn
        Bezeichnungen 
        n Anzahl Zinsperiode
        i Zinsatz
        Sk Schuldsumme nach k Zeiten
        Tk Tilgunsrate  (schuldminderung)
        Zk Zinsbetragh
        Ak = Tk + Zk Annuität (Rückzahlung)

        Fälligketien 
            Tilgunsperiode  = Zindperiode
            Tilgunsperiode  < Zindperiode
            vorschüssig nachschüssig
        Ratentilgun Tilgun kosntatn, Annuität variable
        Annuitätentilgung Annuität konstant, Tilgunsrate variabel

        Ratentilgun 
            Tilgunsperiode  = Zindperiode
            T = s0 / n
            Zk = sk-1 +i
            Tilgungsplan erstellen
            Formel TK = t = s0/n

            SK = s0-k+T = S0 + (1-k/n)
            Zk = i+Sk-1 = i + S0 * (1-(k-1)/n)
            Ak = t+ Zk = s0 / n + i + S0 * (1-(k-1)/n)
        Annutitäentilgung
            Frmel der Rentenrechnugn, R0 = S0, n und i bekannt , gesucht ist r (=A =Ak), bbekannter Barweer, nachschüssige Ratenzahlung
            A = r = (S0+q^n+(q-1)) / (q^n - 1), q = 1+i


        Tilgunsperiode < zinsperiode   
            Ähnliches Vorgehen wie bei Rentenrechnung, Tilungzuahlung wird bis zu selben Periode hochgerechnet wie Zindperiode, dieser wert wird dann verzinst
            re = r+ (m+i/2 + (m-1)) 

            r = r / (m+i/2 + (m-1)) = A / (m+i/2 + (m-1))

\end{document}